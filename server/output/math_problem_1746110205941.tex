\documentclass[a4paper,11pt]{article}

% 日本語対応
\usepackage{xeCJK}
\setCJKmainfont{Yu Gothic}

% 数学関連パッケージ
\usepackage{amsmath}
\usepackage{amssymb}
\usepackage{amsthm}
\usepackage{amsfonts}
\usepackage{mathtools}

% 図形描画用パッケージ
\usepackage{tikz}
\usepackage{pgfplots}
\pgfplotsset{compat=1.18}
\usetikzlibrary{intersections,patterns,angles,quotes,calc,fillbetween}

% その他パッケージ
\usepackage{graphicx}
\usepackage{enumitem}
\usepackage{float}
\usepackage{hyperref}
\usepackage{fancyhdr}
\usepackage{geometry}
\usepackage{xcolor}

% ページ設定
\geometry{
  a4paper,
  top=25mm,
  bottom=25mm,
  left=25mm,
  right=25mm
}

% 数式環境設定
\numberwithin{equation}{section}
\newtheorem{theorem}{定理}
\newtheorem{lemma}[theorem]{補題}
\newtheorem{proposition}[theorem]{命題}
\newtheorem{corollary}[theorem]{系}
\theoremstyle{definition}
\newtheorem{definition}[theorem]{定義}
\newtheorem{example}[theorem]{例}
\newtheorem{exercise}{問題}
\theoremstyle{remark}
\newtheorem*{remark}{注意}
\newtheorem*{solution}{解答}

% ヘッダーとフッターの設定
\pagestyle{fancy}
\fancyhf{}
\fancyhead[L]{数学問題}
\fancyhead[R]{\today}
\fancyfoot[C]{\thepage}
\renewcommand{\headrulewidth}{0.4pt}
\renewcommand{\footrulewidth}{0.4pt}

% タイトル情報
\title{数学問題}
\author{数学問題生成AIシステム}
\date{\today}

% ドキュメント開始
\begin{document}

\maketitle

% 問題セクション
\section*{問題}

% 問題文を挿入(変数で置換)
24と36の最小公倍数を求めるには、まずそれぞれの数を素因数分解します。

\[24 = 2^3 \times 3\]
\[36 = 2^2 \times 3^2\]

最小公倍数は、各素因数の最大の指数を取ります。

\[2^3 \times 3^2 = 8 \times 9 = 72\]

したがって、24と36の最小公倍数は72です。

LaTeX形式での解答は以下の通りです。

\[
\text{最小公倍数} = 2^3 \times 3^2 = 8 \times 9 = 72
\]

% 図形が必要な場合はここに挿入


% 解答セクション(オプショナル - 表示したくない場合はコメントアウト)
\section*{解答}

% 解答を挿入(変数で置換)
72

% 解説セクション(オプショナル - 表示したくない場合はコメントアウト)
\section*{解説}

% 解説を挿入(変数で置換)
まず、$24$と$36$の最大公約数を求めます。$24 = 2^3 \times 3^1$, $36 = 2^2 \times 3^2$ であり、共通する素因数は $2^2$ と $3^1$ です。したがって、最大公約数は $2^2 \times 3^1 = 12$ です。

最小公倍数を求めるには、両数の素因数分解から、それぞれの素因数の最大乗数を取り、乗じ合わせます。$24 = 2^3 \times 3^1$ と $36 = 2^2 \times 3^2$ より、最小公倍数は $2^3 \times 3^2 = 72$ です。

\end{document} 
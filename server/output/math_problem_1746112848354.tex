\documentclass[a4paper,11pt]{article}

% 日本語対応
\usepackage{xeCJK}
\setCJKmainfont{Yu Gothic}

% 数学関連パッケージ
\usepackage{amsmath}
\usepackage{amssymb}
\usepackage{amsthm}
\usepackage{amsfonts}
\usepackage{mathtools}

% 図形描画用パッケージ
\usepackage{tikz}
\usepackage{pgfplots}
\pgfplotsset{compat=1.18}
\usetikzlibrary{intersections,patterns,angles,quotes,calc,fillbetween}

% その他パッケージ
\usepackage{graphicx}
\usepackage{enumitem}
\usepackage{float}
\usepackage{hyperref}
\usepackage{fancyhdr}
\usepackage{geometry}
\usepackage{xcolor}

% ページ設定
\geometry{
  a4paper,
  top=25mm,
  bottom=25mm,
  left=25mm,
  right=25mm
}

% 数式環境設定
\numberwithin{equation}{section}
\newtheorem{theorem}{定理}
\newtheorem{lemma}[theorem]{補題}
\newtheorem{proposition}[theorem]{命題}
\newtheorem{corollary}[theorem]{系}
\theoremstyle{definition}
\newtheorem{definition}[theorem]{定義}
\newtheorem{example}[theorem]{例}
\newtheorem{exercise}{問題}
\theoremstyle{remark}
\newtheorem*{remark}{注意}
\newtheorem*{solution}{解答}

% ヘッダーとフッターの設定
\pagestyle{fancy}
\fancyhf{}
\fancyhead[L]{数学問題}
\fancyhead[R]{\today}
\fancyfoot[C]{\thepage}
\renewcommand{\headrulewidth}{0.4pt}
\renewcommand{\footrulewidth}{0.4pt}

% タイトル情報
\title{数学問題}
\author{数学問題生成AIシステム}
\date{\today}

% ドキュメント開始
\begin{document}

\maketitle

% 問題セクション
\section*{問題}

% 問題文を挿入(変数で置換)
関数 $f(x) = 2x^2 - 8x + 6$ の頂点を求めよ。

% 図形が必要な場合はここに挿入
\begin{figure}[H]
  \centering
  \begin{tikzpicture}
    \begin{axis}[
      axis lines=middle,
      xlabel=$x$,
      ylabel=$y$,
      title={二次関数 $f(x) = 1x^2$},
      xmin=-5, xmax=5,
      ymin=-2, ymax=23,
      xtick={-5,...,5},
      ytick={-5,...,5},
      grid=both,
      grid style={line width=.1pt, draw=gray!10},
      major grid style={line width=.2pt,draw=gray!50},
      axis lines=middle,
      minor tick num=5,
      enlargelimits={abs=0.5},
      axis line style={latex-latex},
      ticklabel style={font=\tiny, fill=white},
      xlabel style={at={(ticklabel* cs:1)}, anchor=north west},
      ylabel style={at={(ticklabel* cs:1)}, anchor=south west}
    ]
    \addplot[thick, blue, domain=-5:5] {1*x^2};
    \addlegendentry{$f(x) = 1x^2$};
    \addplot[mark=*, mark size=3pt, color=red] coordinates {(0,0)};
    \node[anchor=south west] at (axis cs:0,0) {頂点 $(0, 0)$};
    \addplot[mark=*, mark size=2pt, color=blue] coordinates {(0,0)};
    \node[anchor=south west] at (axis cs:0,0) {$x = 0.00$};
    \addplot[mark=*, mark size=2pt, color=green] coordinates {(0,0)};
    \node[anchor=south west] at (axis cs:0,0) {$y = 0$};
    \end{axis}
  \end{tikzpicture}
\end{figure}

% 解答セクション(オプショナル - 表示したくない場合はコメントアウト)
\section*{解答}

% 解答を挿入(変数で置換)
LaTeX形式では、点の座標をそのまま記述するだけで十分です。したがって、入力文 "(2, -2)" をLaTeX形式に変換すると、以下のようになります。

```
$(2, -2)$

% 解説セクション(オプショナル - 表示したくない場合はコメントアウト)
\section*{解説}

% 解説を挿入(変数で置換)
二次関数の頂点の公式を使用して、頂点を求めます。二次関数 $f(x) = ax^2 + bx + c$ の頂点は、$\left(-\frac{b}{2a}, f\left(-\frac{b}{2a}\right)\right)$ になります。ここで、$a = 2$, $b = -8$, $c = 6$ ですから、$x$ 座標は $-\left(-\frac{8}{2 \cdot 2}\right) = 2$ です。次に、この $x = 2$ を関数 $f(x)$ に代入して、$y$ 座標を求めます。$f(2) = 2 \cdot (2)^2 - 8 \cdot (2) + 6 = 8 - 16 + 6 = -2$。よって、頂点は $(2, -2)$ です。

\end{document} 
\documentclass[a4paper,11pt]{article}

% 日本語対応
\usepackage{xeCJK}
\setCJKmainfont{Yu Gothic}

% 数学関連パッケージ
\usepackage{amsmath}
\usepackage{amssymb}
\usepackage{amsthm}
\usepackage{amsfonts}
\usepackage{mathtools}

% 図形描画用パッケージ
\usepackage{tikz}
\usepackage{pgfplots}
\pgfplotsset{compat=1.18}

% その他パッケージ
\usepackage{graphicx}
\usepackage{enumitem}
\usepackage{float}
\usepackage{hyperref}
\usepackage{fancyhdr}
\usepackage{geometry}

% ページ設定
\geometry{
  a4paper,
  top=25mm,
  bottom=25mm,
  left=25mm,
  right=25mm
}

% ヘッダーとフッターの設定
\pagestyle{fancy}
\fancyhf{}
\fancyhead[L]{数学問題}
\fancyhead[R]{\today}
\fancyfoot[C]{\thepage}
\renewcommand{\headrulewidth}{0.4pt}
\renewcommand{\footrulewidth}{0.4pt}

% タイトル情報
\title{数学問題}
\author{数学問題生成AIシステム}
\date{\today}

% ドキュメント開始
\begin{document}

\maketitle

% 問題セクション
\section*{問題}

% 問題文を挿入(変数で置換)
関数 f(x) = $x^2$ のグラフを描き、$x = 1$ における接線の方程式を求めよ。

% 解答セクション(オプショナル - 表示したくない場合はコメントアウト)
\section*{解答}

% 解答を挿入(変数で置換)
$y = 2$x - 1$

% 解説セクション(オプショナル - 表示したくない場合はコメントアウト)
\section*{解説}

% 解説を挿入(変数で置換)
関数 f(x) = $x^2$ の $x = 1$ における微分係数は f'(1) = 2・x|_{$x=1$} = 2 である。よって、点 (1, 1) を通り、傾き 2 の直線の方程式は $y - 1$ = 2($x - 1$) を解いて $y = 2$x - 1$ となる。

% 図形が必要な場合はここに挿入
\begin{figure}[H]
  \centering
  \begin{tikzpicture}
    \begin{axis}[
      axis lines=middle,
      xlabel=$x$,
      ylabel=$y$,
      
      xmin=0, xmax=2,
      ymin=0, ymax=4,
      xtick={-5,...,5},
      ytick={-5,...,5},
      grid=both,
      grid style={line width=.1pt, draw=gray!10},
      major grid style={line width=.2pt,draw=gray!50},
      axis lines=middle,
      minor tick num=5,
      enlargelimits={abs=0.5},
      axis line style={latex-latex},
      ticklabel style={font=\tiny, fill=white},
      xlabel style={at={(ticklabel* cs:1)}, anchor=north west},
      ylabel style={at={(ticklabel* cs:1)}, anchor=south west}
    ]
    \addplot[blue, domain=0:2] {x^2};
    \addlegendentry{f(x) = x^2};
    \addplot[red, domain=0:2] {2*x - 1};
    \addlegendentry{y = 2x - 1};
    \addplot[mark=*, mark size=2pt, color=red] coordinates {(1,1)};
    \end{axis}
  \end{tikzpicture}
\end{figure}

\end{document} 